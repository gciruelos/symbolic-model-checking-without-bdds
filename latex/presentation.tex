\documentclass{beamer}
\usepackage{beamerthemeshadow}
\usepackage[spanish]{babel}
\usepackage[utf8]{inputenc}
\usepackage[T1]{fontenc}
\usepackage{enumerate}
\usepackage{mathtools}
\usepackage{mathpazo}
\usepackage{amssymb}

\newcommand{\until}{\mathrel{\mathbf{U}}}
\newcommand{\rel}{\mathrel{\mathbf{R}}}
\newcommand{\opnext}{\mathbf{X}}
\newcommand{\allp}{\mathbf{A}}
\newcommand{\existsp}{\mathbf{E}}

\begin{document}
\title{Symbolic Model Checking Without BDDs}
\author{AA}
\date{20 de septiembre de 2017}


\frame{\titlepage}

\section{Presentación}
\frame{\frametitle{Presentación}
\begin{itemize}
    \item Symbolic Model Checking withouth BDDs
    \vspace{2em}
    \item Autores
\end{itemize}
}

\section{Prácticas de la materia}
\frame{\frametitle{Presentación}
El model checking es una t\'ecnica para verificar sistemas reactivos dado un sistema de transici\'on que lo represente y una especificaci\'on (usualmente en l\'ogicas temporales).
\bigskip

Symbolic model checking es una t\'ecnica que usa una representaci\'on can\'onica (mediante el uso de OBDDs) de los sistemas de transici\'on y de las f\'ormulas a chequear.
\bigskip

Dicha representaci\'on maneja hasta el orden de $10^{20}$ estados, para sistemas m\'as complejos, el c\'omputo se vuelve intratable.

}

\frame{\frametitle{Presentación}
Sin embargo, los OBDDs son muy sensibles al orden de las variables y adem\'as, en el peor caso su tamanio no es polinomial en la cantidad de variables.
\bigskip

En este trabajo presentan un una t\'ecnica basada en SAT para model checking simb\'olico. La idea es considerar contraejemplos de un cierto tamanio y generar una f\'ormula proposicional que sea satisfacible si y solo si existe tal contraejemplo.
\bigskip

La ventaja es que la reducci\'on se puede hacer en tiempo polinomial y no requiere la construcci\'on del
aut\'omata completo.

El m\'etodo propuesto es muy r\'apido , dada la naturaleza depth-first de SAT, y adem\'as encuentra contraejemplos de longitud m\'inima.
}

\frame{\frametitle{Presentación}
{\bf Definition 1.} Una estructura de Kripke es una tupla $M = (S, I, T, \mathcal{L})$ con un conjunto finito de
estados $S$, un conjunto de estados iniciales $I \subseteq S$, una relaci\'on de transici\'on $T \subseteq S \times S$, y un etiquetado de estados $\mathcal{L} : S \to \mathcal{P}(A)$ de proposiciones at\'omicas $A$.
\bigskip

Aclaraciones
\begin{itemize}
  \item Usamos una codificaci\'on binaria de las variables.
  \item $f_I(s) \iff s \in I$.
  \item $f_p(s) \iff p \in \mathcal{L}(s)$.
  \item $(s, t) \in T \iff s \to t$.
  \item $\pi = (s_0, s_n, ...)$ definimos $\pi_i = s_i$ y $\pi^i = (s_i, s_{i+1}, ...)$.
\end{itemize}
}

\frame{\frametitle{Presentación}
{\bf Definition 2.} Sem\'antica. Dada una estructura de Kripke $M$, un camino $\pi$ en $M$, $f$ una f\'ormula de LTL. Entonces $\pi \models f$ ($f$ es v\'alido a lo largo de $\pi$) se define como sigue.
\bigskip

\begin{itemize}
  \item $\pi \models p \iff p\in \mathcal{L}(\pi(0))$
  \item $\pi \models \neg p \iff p \not\in \mathcal{L}(\pi(0))$
  \item $\pi \models f \land g \iff \pi \models f \text{ y } \pi \models g$
  \item $\pi \models f \lor g \iff \pi \models f \text{ o } \pi \models g$
  \item $\pi \models \square f \iff \forall i. \pi^i \models f$
  \item $\pi \models \lozenge f \iff \exists i. \pi^i \models f$
  \item $\pi \models \opnext f \iff \pi^1 \models f$
  \item $\pi \models f \until g \iff \exists i [ \pi^i \models g \text{ y } \forall j < i. \pi^j \models f]$
  \item $\pi \models f \rel g \iff \forall i [ \pi^i \models g \text{ o } \exists j < i. \pi^j \models f]$
\end{itemize}
}


\frame{\frametitle{Presentación}
{\bf Definition 3 (Validez).} Una f\'ormula LTL $f$ es universalmente v\'alida en una estructura de Kripke $M$ ($M \models \allp f$) si y solo si $\pi \models f$ para todos los caminos $\pi$ en $M$ con $\pi(0) \in I$.
Una f\'ormula LTL $f$ es existencialmente v\'alida en una estructura de Kripke $M$ ($M \models \existsp f$) si y solo si existen un camino $\pi$ en $M$ con $\pi(0) \in I$ tal que $\pi \models f$.

\bigskip
  $M \models \allp f   \iff   M \not\models \existsp \neg f$

% Es claro que una f\'ormula LTL $f$ es universalmente v\'alida en una estructura de Kripke si y solo si su negaci\'on no es existencialmente v\'alida.

\bigskip
Reducimos la b\'usqueda de un contrajemplo de $\allp f$ a la b\'usqueda de una traza que satisfaga $\neg f$.

\bigskip
La idea b\'asica del bouded model checking es considerar solo un prefijo finito de un camino que sea soluci\'on al problema de model checking existencial.
}

\frame{\frametitle{Presentación}
Que consideremos s\'olo prefijos finitos no quiere decir que consideremos s\'olo caminos finitos.

\bigskip
Un prefijo finito puede representar un camino infinito si el prefijo contiene un loop.

% FIGURA 2
}

\frame{\frametitle{Presentación}
% FIGURA 2
{\bf Definici\'on 4.} Para $l \leq k$ llamamos a un camino $\pi$ un $(k,l)$-loop si $\pi(k) \to \pi(l)$
y $\pi = u v^{\omega}$ con $u = (\pi(0), ..., \pi(l \iff 1))$ y $v = (\pi(l), ..., \pi(k))$.
Llamamos a $\pi$ simplemente un $k$-loop si hay un $l \in \mathbb{N}$ tal que $l \leq k$ para el cual $\pi$
es un $(k, l)$-loop.

\bigskip
{\bf Definici\'on 5. (Bounded semantics para trazas con loop)}
Sea $k \in \mathbb{N}$ y sea $\pi$ un $k$-loop y $f$ una f\'ormula LTL.
$\pi_k \models f \iff \pi \models f$.
}

\frame{\frametitle{Presentación}
{\bf Definition 6. (Bounded semantics para trazas sin loops)}
Sea $k \in \mathbb{N}$ y sea $\pi$ una traza que no es un $k$-loop y $f$ una f\'ormula LTL.

$\pi \models_k f \iff \pi \models^0_k f$

\begin{itemize}
  \item $\pi \models^i_k p \iff p\in \mathcal{L}(\pi(i))$
  \item $\pi \models^i_k \neg p \iff p \not\in \mathcal{L}(\pi(i))$
  \item $\pi \models^i_k f \land g \iff \pi \models^i_k f \text{ y } \pi \models^i_k g$
  \item $\pi \models^i_k f \lor g \iff \pi \models^i_k f \text{ o } \pi \models^i_k g$
  \item $\pi \models^i_k \square f \iff false$
  \item $\pi \models^i_k \lozenge f \iff \exists i \leq j \leq k. \pi \models^j_k f$
  \item $\pi \models^i_k \opnext f \iff i < k \text{ y } \pi \models^{i+1}_k f$
  \item {\small $\pi \models^i_k f \until g \iff \exists j, i \leq j \leq k [ \pi \models^j_k g \text{ y } \forall n, i \leq n < j. \pi \models^n_k f]$}
  \item {\small $\pi \models^i_k f \rel g \iff \exists j, i \leq j \leq k [ \pi \models^j_k f \text{ y } \forall n, i \leq n < j. \pi \models^n_k g]$}
\end{itemize}
}


\frame{\frametitle{Presentación}
{\bf Lemma 7. (Soundness / Correctitud)}
Sea $f$ una f\'ormula LTL, $k \in \mathbb{N}$ y $\pi$ un camino $\pi \models_k f \implies \pi \models f$.
\bigskip

{\bf Lemma 8. (Completeness / Completitud para existenciales)}
Sea $f$ una f\'ormula LTL, $M$ estructura de Kripke. Si $M \models \existsp f$ entonces existe
$k \in \mathbb{N}$ tal que $M \models_k \existsp f$.
\bigskip

{\bf Teorema 9. (Correctitud y completitud para  existenciales)}
Sea $f$ una f\'ormula LTL, $M$ estructura de Kripke. $M \models \existsp f$ si y solo si existe
$k \in \mathbb{N}$ tal que $M \models_k \existsp f$.
}



\frame{
\huge
\begin{center}
¿Preguntas?
\end{center}
}

\end{document}


